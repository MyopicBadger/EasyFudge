\documentclass[10pt,a4paper,oneside]{article}
\usepackage[utf8]{inputenc}
\usepackage{amsmath}
\usepackage{amsfonts}
\usepackage{amssymb}
\usepackage{graphicx}
\usepackage[left=2cm,right=2cm,top=2.5cm,bottom=2.5cm]{geometry}
\author{Graeme Rockhill}
\title{Easy Chocolate Fudge}
\begin{document}
\maketitle
\section{Ingredients}
\begin{itemize}

\item 400g dark or milk chocolate (I use Galaxy)
\item 397g can condensed milk (this is a standard size, for some reason)
\item 25g butter
\item 175g icing sugar (plus approx. 25g for dusting)
\item A splash of vanilla (optional)

\end{itemize}
\section{Implements}
\begin{itemize}

\item Baking tray/tin (the one I have is about A4-sized, and about a centimetre deep)
\item Baking/greaseproof paper
\item A sieve (technically optional, and slightly less messy if you don't use it, but the consistency of the fudge suffers without it)
\item A mixing implement (a wooden spoon does the job nicely, an electric whisk would be less effort)
\item A saucepan
\item A sharp knife or a pizza cutter
\item A set of scales

\end{itemize}
\section{Instructions}
\begin{enumerate}
\item Break or chop the chocolate into small chunks. If the chocolate has been frozen beforehand (which I advocate) this can easily be achieved by dropping it on a tiled surface, then whacking it against the worktop to break up any larger pieces that survived the initial impact intact.

\item Place the chocolate in a non-stick saucepan with the condensed milk and butter.

\item Melt the ingredients together, stirring frequently until smooth and silky. Note that you should begin stirring very quickly after placing the ingredients in the pan. If this is left too long, the bottom layer of the ingredients will overcook. This can happen faster than you expect, and will leave gritty bits of caramelised chocolate.

\item Sieve in the icing sugar and mix thoroughly (you can use an electric whisk if you like, although I use a wooden spoon).

\item Pour fudge into a tin that has been lined with baking paper and dusted with icing sugar.

\item Chill in the fridge for until set, then cut into squares. You can do this using a sharp knife, although you'll find it much easier using a pizza cutter. As a bonus, this will produce crisper edges.

\item Store in an airtight container in the fridge for up to 2 weeks.
\end{enumerate}

One thing it would be interesting to try would be making fudge from white chocolate. The original recipe advocates using chocolate with greater than forty percent coco, Galaxy chocolate is already lower than that.

\end{document}